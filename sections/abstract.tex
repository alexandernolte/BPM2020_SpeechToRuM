\begin{abstract}
In recent years declarative, constraint-based approaches have been proposed to model loosely-structured business processes, mediating between support and flexibility. A notable example is the \Declare\ framework, equipped with a graphical declarative language whose semantics can be characterized with several logic-based formalisms. Up to now, the main problem hampering the use of \Declare\ constraints in practice has been the difficulty with modeling constraints either with a formal notation, difficult to understand for users without a background in temporal logics, or with their graphical notation that has been proved to be unintuitive. In this work, we present and evaluate an analysis toolkit that tries to bypass this issue by providing the user with the possibility of modeling \Declare\ constraints using her/his own way of expressing them. The toolkit contains a \Declare\ modeler equipped with a speech recognition mechanism that takes as input a vocal statement from the user and converts it into the closest (set of) \Declare\ constraint(s). The constraints that can be modeled with the tool cover the entire Multi-Perspective extension of \Declare\ (\MPDeclare) that can not only express control-flow properties, but also other perspectives like data and time. Even if we consider \Declare\ as use case, the tool/evaluation presented in this paper can be considered, to the best of our knowledge, the first attempt to test the usability of speech recognition in business process modeling.
\end{abstract} 