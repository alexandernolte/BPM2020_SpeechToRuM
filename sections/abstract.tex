\begin{abstract}
Declarative process models define the behaviour of business processes as a set of constraints. Declarative process discovery aims at inferring such constraints from event logs. Existing discovery techniques verify the satisfaction of candidate constraints over the log, but completely neglect their interactions. As a result, the inferred constraints can be mutually contradicting and their interplay may lead to an inconsistent process model that does not accept any trace. In such a case, the output turns out to be unusable for enactment, simulation or verification purposes. In addition, the discovered model contains, in general, redundancies that are due to complex interactions of several constraints and that cannot be solved using existing pruning approaches.
We address these problems by proposing a technique that automatically resolves conflicts within the discovered models and is more powerful than existing pruning techniques to eliminate redundancies. First, we formally define the problems of constraint redundancy and conflict resolution. Thereafter, we introduce techniques based on the notion of an \emph{automata-product monoid} that guarantee the consistency of the discovered models and, at the same time, keep the most interesting constraints in the pruned set. We evaluate the devised techniques on real-world benchmarks.
\end{abstract} 