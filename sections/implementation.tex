\section{The Tool}
\label{sec:implementation}
The proposed approach was implemented in RuM, a modeling and analysis toolkit for Rule Mining.\footnote{The tool can be found at \url{https://tinyurl.com/r3zm45f}} The approach presented in this paper is the modeling component of the tool called Speech2RuM. In \figurename~\ref{fig:toolNew}, a screenshot of the latest version of the tool is shown. In the top-left area of the screen the recognized vocal input is shown. It is possible to record a constraint (control-flow perspective) as shown in \figurename~\ref{fig:controlflow} where an Init and a Response constraint are modeled. After having inserted a standard \Declare\ constraint, the user can record an activation, a correlation and a time condition as shown in \figurename~\ref{fig:toolNew} where an activation (\emph{amount} is greater than 100 and \emph{customer type} is \emph{gold}) and a correlation (activation and target share the same value for attribute \emph{amount}) condition have already been recorded and a time condition is extracted from the vocal input \textit{``not before 2 hours and within 12 hours''}. The constraints expressed with their graphical notation are shown in the top-right area of the screen, and modifiable lists of activities and constraints are shown in the bottom-left and bottom-right areas of the screen.

\begin{figure}[t!]
	\includegraphics[width=\textwidth]{figures/toolAfter}
	\caption{Screenshot of the tool}
	\label{fig:toolNew}
\end{figure}


\begin{figure}[t!]
  \centering
  \subfloat[Inserting an Init constraint \label{fig:init}]{
    \includegraphics[scale = 0.4]{figures/init}
  }\hspace{0.2cm}
  \subfloat[Inserting a Response constraint \label{fig:response}]{
    \includegraphics[scale = 0.4]{figures/response}
  }
\caption{Inserting control-flow properties.}
\label{fig:controlflow}
\end{figure}

