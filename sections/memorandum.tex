% !TeX root = ../Main.tex
%
\newglossaryentry{test}{name={test},description={this is a test}}
\newacronym{testacro}{TA}{Test Acronym}
%
\section*{Memorandum}
\label{sec:memo}
Tips.
%
For your comments, please use
\todo[inline]{\texttt{todo} environment, either inline or }
or \todo{aside}.
%
For enumerated inline lists, please use
\begin{inparaenum}[\itshape(i)\upshape]
\item the
\item \texttt{inparaenum}
\item environment.
\end{inparaenum}
%
For specific terms, which may change in the future, add a glossary entry in \texttt{Addons/defs.tex},
and use
\begin{inparaenum}[\itshape(i)\upshape]
\item the \texttt{gls} command to use them (e.g., \gls{test}),
\item the \texttt{Gls} command for capitalising them (e.g., \Gls{test}),
\item the \texttt{glspl} command for obtaining their plural form (e.g., \glspl{test}), and
\item the \texttt{Glspl} command for for obtaining their plural capitalised form (e.g., \Glspl{test}).
\end{inparaenum}
%
For acronyms, they are meant be defined within the same file as follows: \texttt{{\textbackslash}newacronym\{testacro\}\{TA\}\{Test Acronym\}}.
They can be referenced by using \texttt{{\textbackslash}Gls} (plus variants for plural forms etc.).
Result is: you get the full version the first time the acronym is used
-- \texttt{{\textbackslash}gls\{testacro\}} is displayed as \Gls{testacro}.
From the second time onwards, the call will return the acronym only, without any description:
\texttt{{\textbackslash}gls\{testacro\}} is displayed as \Gls{testacro}.
To force the representation to have either long or short versions, use
\texttt{{\textbackslash}acrlong\{testacro\}}
(leading to \acrlong{testacro})
or
\texttt{{\textbackslash}acrshort\{testacro\}}
(displaying \acrshort{testacro}).
%
For referring to sections, tables, figures, etc., use the \texttt{Cref} command (e.g., \Cref{sec:memo}).
\texttt{Cref} capitalises the specifier of the referred document part (e.g., \Cref{sec:memo}), whereas \texttt{cref}. leaves the first letter lowercase (e.g., \cref{sec:memo}).
%