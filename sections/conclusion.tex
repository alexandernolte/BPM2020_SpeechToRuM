% !TeX root = ../Main.tex
\section{Conclusion}
\label{sec:conclusion}
%
In this paper, we addressed the problems of redundant and inconsistent constraint sets that are potentially generated by declarative process mining tools. We formalised the problem based on the notion of automata-product monoid and devised the corresponding analysis algorithms. The evaluation based on our prototypical implementation shows that typical constraint sets can be further pruned such that the result is consistent and locally minimal.
Our contribution complements research on declarative process execution and simulation and provides the basis for a fair comparison of procedural and declarative representations.

In future research, we aim at extending our work towards other perspectives of processes. When mining declarative constraints with references to data and resources, one of the challenges will be to identify comparable notions of subsumption and causes of inconsistency. We also plan to follow up on experimental research comparing Petri nets and \gls{declare}. The notions defined in this paper help design declarative and procedural process models that are equally consistent and minimal, such that an unbiased comparison would be feasible. 