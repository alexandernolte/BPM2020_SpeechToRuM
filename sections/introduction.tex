% !TeX root = ../Main.tex
% !TeX spellcheck = en_US

\section{Introduction}
\label{sec:introduction}

Process models are an important means to capture information on organizational processes, serving as a basis for communication and often as the starting point for analysis and improvement~\cite{dumas2013fundamentals}.
For processes that are relatively structured, \emph{imperative} process modeling notations, such as the Business Process Model and Notation (BPMN), are most commonly employed. However, other processes, in particular knowledge-intensive ones, are more flexible and, therefore, less structured. An important characteristic of such processes is that it is typically infeasible to specify the entire spectrum of allowed execution orders in advance~\cite{dicicio2015knowledge}, which severely limits the applicability of the imperative process modeling paradigm. Instead, such processes are better captured using \emph{declarative} process models defined in process modeling languages like \Declare\ whose semantics
can be characterized using temporal logic properties. These models, indeed, do not require an explicit definition of  all allowed execution orders, but rather use constraints to define the boundaries of the permissible process behavior \cite{vanderaalst2009declarative}.

Although their benefits are apparent, establishing declarative process models is known to be difficult, especially for domain experts that generally lack expertise in temporal logics, and in most of the cases find the graphical notation of \Declare\ constraints unintuitive.
Due to these barriers to declarative model creation, a preliminary approach has been presented in~\cite{vanderaa2019extracting} that automatically extracts declarative process models from natural language texts, similarly to what many works focusing on the generation of imperative process models from natural language descriptions have largely investigated (cf., \cite{friedrich2011process,schumacher2013extracting,vanderaa2018checking,de2017assisting}).
%processes are often documented using natural language instead~\cite{}.  Recognizing this, several works have been developed that automatically extract process models from natural language texts (cf., \cite{friedrich2011process,schumacher2013extracting,vanderaa2018checking,de2017assisting}).
%The vast majority of these focus on the generation of imperative process descriptions, whereas only one preliminary approach~\cite{vanderaa2019extracting} targets the extraction of declarative process models.

%In this work, we present and evaluate a tool that tries to bypass this issue by providing the user with the possibility of modeling \Declare\ constraints using his/her own way of expressing them. The tool is a \Declare\ modeler equipped with a vocal assistant that takes as input a vocal statement from the user and converts it into the closest (set of) \Declare\ constraint(s). The inputs that can be provided to the tool cover the entire Multi-Perspective extension of \Declare\ (MP-\Declare\) that can not only express control-flow properties, but also other perspectives like data and time. Even if we consider \Declare\ as use case, the tool/evaluation presented in this paper can be considered, to the best of our knowledge, the first attempt to test the usability of speech recognition in business process modeling.

%we aim to support the elicitation of declarative process models based on natural language input. In particular,

In this work, we go beyond the extraction of \Declare\ constraints from natural language descriptions, and present and evaluate an interactive approach that takes as input a vocal statement from the user and employs speech recognition to convert it into the closest (set of) \Declare\ constraint(s). The approach has been integrated in a declarative modeling and analysis toolkit. With this tool, the user is not required to have any experience in temporal logics nor to be familiar with the graphical notation of \Declare\ constraints, but can express temporal properties using his/her own words. The temporal properties that can be modeled with the tool cover the entire Multi-Perspective Declare (\MPDeclare) language that can not only express control-flow properties, but also other perspectives like data, time and resource. An important contribution of the paper is that the evaluation presented in this paper can be considered, to the best of our knowledge, the first attempt to test the usability of speech recognition in business process modeling.

\todoinline{describe evaluation results}

To summarize, we are able to provide the following contributions in comparison to the state of the art:
\begin{compactenum}
	\item we greatly increase the coverage of \Declare\ constraint types wrt.~\cite{vanderaa2019extracting};
	\item we handle more flexible inputs so that more expressions can be recognized by the speech recognition component;
    \item constraints are now linked to each other, enabling the definition of actual declarative process models, rather than just individual constraints;	
    \item no one-shot approach;  \todo{what does this mean?}
	\item we ensure additional expressiveness by incorporating conditions from data, time, and resource perspectives
	\item the generated process models can be directly used as input for, e.g., conformance checking, log generation, and process monitoring (being the modeler integrated in an analysis toolkit);
    \item we test the usability of speech recognition in business process modeling.
\end{compactenum}


%- finally, all of the above is contained in a tool that uses speech recognition for the efficient elicitation of constraints, employs interaction to resolve ambiguity and enable step-by-step constraint enrichment, both its intermediatry and final output are directly available as part of a state-of-the-art declarative modeling toolkit (RuM). what can the user then do with these models because of this?


\todoinline{describe remainder of paper}





