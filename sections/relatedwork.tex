% !TeX root = ../Main.tex
\section{Related work}
\label{sec:relatedwork}

Organizations recognize the benefit of using textual documents to capture process specifications~\cite{vanderaa2017fragmentation,selway2015formalising}, given that these can be created and understood by virtually everyone~\cite{friedrich2011process}.
In order to allow these documents to serve as a basis for automated process analysis, such as conformance checking, a variety of techniques have been developed to extract process models from texts.
In the context of imperative process models, 
the technique by Friedrich et al.~\cite{friedrich2011process} is regarded as the state-of-the-art in this context~\cite{riefer2016mining}, though alternatives exist that target other types of input formats, such as group stories~\cite{gonccalves2009business}, and methodological descriptions~\cite{epure2015automatic}.
Other works analyze textual process specifications for model verification~\cite{vanderaa2017comparing,sanchez2017aligning} or directly for process analysis~\cite{vanderaa2018checking,sanchez2017aligning}.
In the context of declarative process models, some recent works also provide support for the extraction of DCR graphs from textual descriptions~\cite{lopez2019assisted,lopez2018process}, whereas preliminary work on the extraction of \Declare\ constraints from texts represents a foundation for our work.

 


