% !TeX spellcheck = en_US
% !TeX program = pdflatex
\documentclass{llncs}
%
%%%%%%%%%%%%%%%%%%%%%%%%%%%%%%%%%%%%%%%%%%%%%%%%%%%%%%%%%%%%%%%%
% Package loading
%%%%%%%%%%%%%%%%%%%%%%%%%%%%%%%%%%%%%%%%%%%%%%%%%%%%%%%%%%%%%%%%
%
\usepackage[official]{eurosym}
\usepackage{amstext} % for \text
\DeclareRobustCommand{\officialeuro}{%
  \ifmmode\expandafter\text\fi
  {\fontencoding{U}\fontfamily{eurosym}\selectfont e}}

% Allows for index generation
\usepackage{makeidx}
% For language-specific hyphenations etc.
%\usepackage[british,english]{babel}
% To determine characteristics of float environments
\usepackage{floatrow}
% For subfigures
\usepackage{subfig}
% For nice links
\usepackage{url}
% For playing with colors in tabular environments
\usepackage{colortbl}
% For math symbols, such as \nexists
\usepackage{amssymb}
% For advanced graphics
\usepackage{graphicx}
% For equations, arrays of equations, defining operator names, etc.
\usepackage{amsmath}
% For cursive math
\usepackage{mathrsfs}
% For enumerating the line numbers
\usepackage[left]{lineno}
% For side notes, missing figures and inline to-do's
\usepackage[textsize=scriptsize,colorinlistoftodos]{todonotes}
% For specifying kewords and acronyms
\usepackage[nonumberlist,acronym,sanitize=none]{glossaries}
\glsdisablehyper
% For commenting out some parts of the text
\usepackage{comment}
% For building hyperlinks
\usepackage[hidelinks]{hyperref}
% For smart references
%\usepackage{cleveref}
%\crefname{algocf}{alg.}{algs.}
%\Crefname{algocf}{Algorithm}{Algorithms}
% TikZ/Pgf advanced graphics
\usepackage{tkz-base}
\usetikzlibrary{decorations.pathmorphing,trees}
% To use inline and other fancy list-like environments (e.g., inparaenum)
\usepackage{paralist}
% To divide a text line into multiple columns
\usepackage{multicol}
% To create good-looking book-style tables
\usepackage{booktabs}
% To play around with list environments
\usepackage{enumitem}
% To create multirow cells in tables
\usepackage{multirow}
% To create rotated cells in tables
\usepackage{rotating}
% To make algorithmic nice-looking pseudocode
\usepackage[ruled,linesnumbered,algo2e]{algorithm2e}
% For strike-through cancellations
\usepackage[normalem]{ulem}
% To properly order references
\usepackage{cite}
\usepackage{tabularx}

%
%%%%%%%%%%%%%%%%%%%%%%%%%%%%%%%%%%%%%%%%%%%%%%%%%%%%%%%%%%%%%%%%
% LaTeX handling, dirty tricks, and more
%%%%%%%%%%%%%%%%%%%%%%%%%%%%%%%%%%%%%%%%%%%%%%%%%%%%%%%%%%%%%%%%
%
\input{addons/commands}
%
%%%%%%%%%%%%%%%%%%%%%%%%%%%%%%%%%%%%%%%%%%%%%%%%%%%%%%%%%%%%%%%%
% Used keywords, symbols, and typesetting commands
%%%%%%%%%%%%%%%%%%%%%%%%%%%%%%%%%%%%%%%%%%%%%%%%%%%%%%%%%%%%%%%%
%
%
%%%%%%%%%%%%%%%%%%%%%%%%%%%%%%%%%%%%%%%%%%%%%%%%%%%%%%%%%%%%%%%%
% The problem
%%%%%%%%%%%%%%%%%%%%%%%%%%%%%%%%%%%%%%%%%%%%%%%%%%%%%%%%%%%%%%%%
%
% Constraint-conflict resolution
%
\newglossaryentry{confliresproblem}{%
	name={non-redundant consistent constraints set},description={The problem of finding the conflicting constraints in a \glsentrytext{declamodel}}}
\newglossaryentry{autopromo}{%
	name={automata-product monoid},description={The algebraic structure in which we find a solution for the \glsentrytext{confliresproblem}}}
%
%%%%%%%%%%%%%%%%%%%%%%%%%%%%%%%%%%%%%%%%%%%%%%%%%%%%%%%%%%%%%%%%
% Declare, Declare mining, etc.
%%%%%%%%%%%%%%%%%%%%%%%%%%%%%%%%%%%%%%%%%%%%%%%%%%%%%%%%%%%%%%%%
%
% Declare
%
\newglossaryentry{declare}{%
	name={\textsc{Declare}},description={A declarative process modelling language and notation}}
%
% Task
%
\newglossaryentry{task}{%
	name={task},description={The non-divisible, elementary activity}}
\def\paramx {\ensuremath{x}}
\def\paramy {\ensuremath{y}}
\def\paramz {\ensuremath{z}}
\newcommand{\taskize}[1] {\ensuremath{\scalebox{0.85}{\textsf{#1}}}}
\def\taska {\taskize{a}}
\def\taskb {\taskize{b}}
\def\taskc {\taskize{c}}
\def\taskd {\taskize{d}}
%
% Process model
%
\newglossaryentry{promod}{%
	name={process model},description={The model of a process}
}
%
% Declarative process model
%
\def\declamodel {\ensuremath{M}}
\newglossaryentry{declamodel}{%
	name={declarative \glsentrytext{promod}},description={The model of a process, expressed by means of constraints},
	symbol={\declamodel}
}
%
% MINERful
%	
\newglossaryentry{minerful}{%
	name={MINERful},description={The declarative process discovery algorithm \glsentrytext{minerful}}}
%
% Declare Maps Miner
%
\newglossaryentry{decmapmin}{%
	name={Declare Maps Miner},description={The declarative process discovery algorithm \glsentrytext{decmapmin}}}
%
% log alphabet
%
\def\logalph {\ensuremath{\mathfrak{A}}}
\newglossaryentry{logalph}{
	name={log alphabet},description={process alphabet, as reflected in the log},%
	symbol={\logalph}}
%
% event
%
\def\evt {\ensuremath{e}}
\newglossaryentry{evt}{
	name={event},description={a record of an instantaneous fact during the process enactment},%
	symbol={\evt}}
%
% trace
%
\def\evttrace {\ensuremath{\vec{t}}}
\newglossaryentry{evttrace}{
	name={trace},description={a sequence of \glsentrytext{event}s},%
	symbol={\evttrace}}
%
% event log
%
\def\evtlog {\ensuremath{L}}
\newglossaryentry{evtlog}{
	name={event log},description={a collection of \glsentrytext{trace}s},%
	symbol={\evtlog}}
%
% subsumption
%
\def\subsum {\ensuremath{\sqsubseteq}}
\newglossaryentry{subsum}{%
	name={subsumption},description={is subsumed by},%
	symbol={\subsum}}
%
% relaxation operator
%
%---\def\relaxop {\ensuremath{\overset{\triangle}{\mathrm{r}}}}
\def\relaxop {\mathscr{R}}
\newglossaryentry{relaxop}{%
	name={relaxation},description={relaxation operator, climbing the \glsentrytext{subsum} hierarchy}}
\newcommand{\relaxopFunc}[1] {\ensuremath{\relaxop\left(#1\right)}}
%
% activation
%
\newglossaryentry{acti}{%
	name={activation},description={activation of a constraint}}
%
% target
%
\newglossaryentry{target}{%
	name={target},description={target}}
%
% constraint
%
\def\constraint {\ensuremath{C}}
\newglossaryentry{con}{%
	name={constraint},description={a temporal business rule},
	symbol={\constraint}
}
\def\constraintPrime {\ensuremath{\constraint'}}
\def\constraintSecond {\ensuremath{\constraint''}}
%
% existence
%
\newglossaryentry{exi}{
	name={existence},
	description={constrains single activities}
}
%
% existence constraint
%
\newglossaryentry{exicon}{
	name={\glsentrytext{exi} \glsentrytext{con}},
	description={constrains single activities}
}
%
% position constraint
%
\newglossaryentry{posicon}{
	name={position \glsentrytext{con}},
	description={constrains the position of activities}
}
%
% cardinality constraint
%
\newglossaryentry{cardicon}{
	name={cardinality \glsentrytext{con}},
	description={limits the number of activities}
}
%
% relation
%
\newglossaryentry{rela}{
	name={relation},
	description={constraint on pairs of activities}
}
%
% relation constraint
%
\newglossaryentry{relacon}{
	name={\glsentrytext{rela} \glsentrytext{con}},
	description={constraint on pairs of activities}
}
%
% unidirectional relation constraint
%
\newglossaryentry{unirelacon}{
	name={unidirectional \glsentrytext{relacon}},
	description={constraint on pairs of activities, out of which one is the activation, as the other is the target}
}
%
% unidirectional forward relation constraint
%
\newglossaryentry{unifwrelacon}{
	name={\glsentrytext{fw}-\glsentrytext{unirelacon}},
	description={constraint on pairs of activities, having the first parameter as the activation, and the second one as the target}
}
\def\fw {\ensuremath{\mathit{fw}}}
\newglossaryentry{fw}{
	name={forward},
	description={forward constraint},
	symbol={\fw}
}
\newcommand{\fwFunc}[1] {\ensuremath{\fw\left(#1\right)}}
%
% unidirectional backward relation constraint
%
\newglossaryentry{unibwrelacon}{
	name={\glsentrytext{bw}-\glsentrytext{unirelacon}},
	description={constraint on pairs of activities, having the second parameter as the activation, and the first one as the target}
}
\def\bw {\ensuremath{\mathit{bw}}}
\newglossaryentry{bw}{
	name={backward},
	description={backward constraint},
	symbol={\bw}
}
\newcommand{\bwFunc}[1] {\ensuremath{\bw(#1)}}
%
% coupling relation constraint
%
\newglossaryentry{corelacon}{
	name={coupling \glsentrytext{con}},
	description={constraint based on pairs of relation constraints}
}
%
% negative relation constraint
%
\newglossaryentry{nega}{
	name={negative},
	description={of a constraint, that negates a coupling relation constraint}
}
\newglossaryentry{negacon}{
	name={\glsentrytext{nega} \glsentrytext{con}},
	description={constraint negating a coupling relation constraint}
}
%
% constraint template
%
\def\contemp {\ensuremath{\mathcal{C}}}
\newglossaryentry{contemp}{%
	name={\glsentrytext{temp}},description={the template of a \glsentrydesc{con}},
	symbol={\contemp}}
\newglossaryentry{temp}{%
	name={template},description={the template of a \glsentrydesc{con}},
	symbol={\contemp}}
\def\contempPrime {\ensuremath{\contemp'}}
\def\contempSecond {\ensuremath{\contemp''}}
\newcommand{\contempFunc}[2] {\ensuremath{\contemp(#1,#2)}}
\newcommand{\contempPrimeFunc}[2] {\ensuremath{\contempPrime(#1,#2)}}

%
% relation constraint template
%
\def\relacontemp {\ensuremath{\mathcal{R}}}
\newglossaryentry{relacontemp}{%
	name={relation template},description={the template of a relation \glsentrydesc{con}},
	symbol={\relacontemp}}
\newcommand{\relacontempFunc}[2] {\ensuremath{\relacontemp(#1,#2)}}
\newcommand{\relacontempPrimeFunc}[2] {\ensuremath{\relacontemp'(#1,#2)}}
\newcommand{\relacontempSecondFunc}[2] {\ensuremath{\relacontemp''(#1,#2)}}
%
% existence constraint template
%
\def\exicontemp {\ensuremath{\mathcal{E}}}
\newglossaryentry{exicontemp}{%
	name={existence template},description={the template of an existence \glsentrydesc{con}},
		symbol={\exicontemp}}
\newcommand{\exicontempFunc}[1] {\ensuremath{\exicontemp(#1)}}
\newcommand{\exicontempPrimeFunc}[1] {\ensuremath{\exicontemp'(#1)}}
\newcommand{\exicontempSecondFunc}[1] {\ensuremath{\exicontemp''(#1)}}
%
% support
%
\def\support {\ensuremath{\mathit{supp}}}
\newglossaryentry{support}{%
	name={support},description={the support of a \glsentrydesc{con}},
	symbol={\support}}
\newcommand{\supportFunc}[1] { \ensuremath{\support \left( #1 \right)} }
\def\tsupp {\ensuremath{\tau_{\textrm{supp}}}}
%
% confidence
%
\def\conf {\ensuremath{\mathit{conf}}}
\newglossaryentry{conf}{%
	name={confidence},description={the confidence level of a \glsentrydesc{con}},
	symbol={\conf}}
\newcommand{\confFunc}[1] { \ensuremath{\conf \left( #1 \right)} }
%
% interest factor
%
\def\intf {\ensuremath{\mathit{IF}}}
\newglossaryentry{intf}{%
	name={interest factor},description={the interest factor of a \glsentrydesc{con}},
	symbol={\intf}}
\newcommand{\intfFunc}[1] { \ensuremath{\intf \left( #1 \right)} }
%
%
%
\def\PartTxt {Participation}
\def\UniqTxt {AtMostOne}
\def\InitTxt {Init}
\def\EndTxt {End}
\def\ResExTxt {RespondedExistence}
\def\RespTxt {Response}
\def\AltRespTxt {AlternateResponse}
\def\ChaRespTxt {ChainResponse}
\def\PrecTxt {Precedence}
\def\AltPrecTxt {AlternatePrecedence}
\def\ChaPrecTxt {ChainPrecedence}
\def\CoExiTxt {CoExistence}
\def\SuccTxt {Succession}
\def\AltSuccTxt {AlternateSuccession}
\def\ChaSuccTxt {ChainSuccession}
\def\NotCoExiTxt {NotCoExistence}
\def\NotSuccTxt {NotSuccession}
\def\NotChaSuccTxt {NotChainSuccession}
%
\def\PartTmp {\ensuremath{\mathit{\PartTxt}}}
\def\UniqTmp {\ensuremath{\mathit{\UniqTxt}}}
\def\InitTmp {\ensuremath{\mathit{\InitTxt}}}
\def\EndTmp {\ensuremath{\mathit{\EndTxt}}}
\def\ResExTmp {\ensuremath{\mathit{\ResExTxt}}}
\def\RespTmp {\ensuremath{\mathit{\RespTxt}}}
\def\AltRespTmp {\ensuremath{\mathit{\AltRespTxt}}}
\def\ChaRespTmp {\ensuremath{\mathit{\ChaRespTxt}}}
\def\PrecTmp {\ensuremath{\mathit{\PrecTxt}}}
\def\AltPrecTmp {\ensuremath{\mathit{\AltPrecTxt}}}
\def\ChaPrecTmp {\ensuremath{\mathit{\ChaPrecTxt}}}
\def\CoExiTmp {\ensuremath{\mathit{\CoExiTxt}}}
\def\SuccTmp {\ensuremath{\mathit{\SuccTxt}}}
\def\AltSuccTmp {\ensuremath{\mathit{\AltSuccTxt}}}
\def\ChaSuccTmp {\ensuremath{\mathit{\ChaSuccTxt}}}
\def\NotCoExiTmp {\ensuremath{\mathit{\NotCoExiTxt}}}
\def\NotSuccTmp {\ensuremath{\mathit{\NotSuccTxt}}}
\def\NotChaSuccTmp {\ensuremath{\mathit{\NotChaSuccTxt}}}
%
\newcommand{\Part}[1] {\ensuremath{\mathit{\PartTxt}(#1)}}
\newcommand{\Uniq}[1] {\ensuremath{\mathit{\UniqTxt}(#1)}}
\newcommand{\Init}[1] {\ensuremath{\mathit{\InitTxt}(#1)}}
\newcommand{\End}[1] {\ensuremath{\mathit{\EndTxt}(#1)}}
\newcommand{\ResEx}[2] {\ensuremath{\mathit{\ResExTxt}(#1,#2)}}
\newcommand{\Resp}[2] {\ensuremath{\mathit{\RespTxt}(#1,#2)}}
\newcommand{\AltResp}[2] {\ensuremath{\mathit{\AltRespTxt}(#1,#2)}}
\newcommand{\ChaResp}[2] {\ensuremath{\mathit{\ChaRespTxt}(#1,#2)}}
\newcommand{\Prec}[2] {\ensuremath{{\mathit{\PrecTxt}}(#1,#2)}}
\newcommand{\AltPrec}[2] {\ensuremath{\mathit{\AltPrecTxt}(#1,#2)}}
\newcommand{\ChaPrec}[2] {\ensuremath{\mathit{\ChaPrecTxt}(#1,#2)}}
\newcommand{\CoExi}[2] {\ensuremath{\mathit{\CoExiTxt}(#1,#2)}}
\newcommand{\Succ}[2] {\ensuremath{\mathit{\SuccTxt}(#1,#2)}}
\newcommand{\AltSucc}[2] {\ensuremath{\mathit{\AltSuccTxt}(#1,#2)}}
\newcommand{\ChaSucc}[2] {\ensuremath{\mathit{\ChaSuccTxt}(#1,#2)}}
\newcommand{\NotCoExi}[2] {\ensuremath{\mathit{\NotCoExiTxt}(#1,#2)}}
\newcommand{\NotSucc}[2] {\ensuremath{\mathit{\NotSuccTxt}(#1,#2)}}
\newcommand{\NotChaSucc}[2] {\ensuremath{\mathit{\NotChaSuccTxt}(#1,#2)}}
%
%
%%%%%%%%%%%%%%%%%%%%%%%%%%%%%%%%%%%%%%%%%%%%%%%%%%%%%%%%%%%%%%%%
% LTL
%%%%%%%%%%%%%%%%%%%%%%%%%%%%%%%%%%%%%%%%%%%%%%%%%%%%%%%%%%%%%%%%
%
\def\LTLnext {\ensuremath{\bigcirc}}
\def\LTLfut {\ensuremath{\Diamond}}
\def\LTLglob {\ensuremath{\Box}}
\def\LTLuntil {\ensuremath{\;\mathcal{U}\,}}
\def\LTLwntil {\ensuremath{\;\mathcal{W}\,}}
%
%
%% Not-sign in Regular Expressions
%
\def\nore{{\textasciicircum}} 
%

\newcommand{\mypar}[1]{\smallskip\noindent\textbf{#1.}}
\newcommand{\todoinline}[1]{\textcolor{red}{\bf {#1}}}
\renewcommand{\sectionautorefname}{Section}
\renewcommand{\subsectionautorefname}{Section}
\renewcommand{\figureautorefname}{Fig.}
%\renewcommand{\tableautorefname}{Tab.}
\newcommand{\lnext}{\ensuremath{\mathbf{X}}}
\newcommand{\lwnext}{\ensuremath{\mathbf{\bar{X}}}}
\newcommand{\luntil}{\ensuremath{\mathbf{U}}}
\newcommand{\lsince}{\ensuremath{\mathbf{S}}}
\newcommand{\lrelease}{\ensuremath{\mathbf{R}}}
\newcommand{\lwuntil}{\ensuremath{\mathbf{W}}}
\newcommand{\lglobally}{\ensuremath{\mathbf{G}}}
\newcommand{\lfuture}{\ensuremath{\mathbf{F}}}
\newcommand{\tnext}{\ensuremath{\mathbf{X}_{I}}}
\newcommand{\twnext}{\ensuremath{\mathbf{\bar{X}_I}}}
\newcommand{\tuntil}{\ensuremath{\mathbf{U}_{I}}}
\newcommand{\tsince}{\ensuremath{\mathbf{S}_{I}}}
\newcommand{\trelease}{\ensuremath{\mathbf{R}_{I}}}
\newcommand{\tglobally}{\ensuremath{\mathbf{G}_{I}}}
\newcommand{\lonce}{\ensuremath{\mathbf{O}}}
\newcommand{\tonce}{\ensuremath{\mathbf{O}_{I}}}
\newcommand{\lyesterday}{\ensuremath{\mathbf{Y}}}
\newcommand{\tyesterday}{\ensuremath{\mathbf{Y}_{I}}}
\newcommand{\lhistorically}{\ensuremath{\mathbf{H}}}
\newcommand{\thistorically}{\ensuremath{\mathbf{H}_{I}}}
\newcommand{\tfuture}{\ensuremath{\mathbf{F}_{I}}}
\newcommand{\responseSA}{{\sc Response(Submit Loan Application, Assess Application)}}
\newcommand{\responseAS}{{\sc Response(Assess Application, Submit Loan Application)}}
\newcommand{\responsesa}{{\sc Response(S,A)}}
\newcommand{\responseas}{{\sc Response(A,S)}}
\newcommand{\response}{{\sc Response}}
\newcommand{\MPDeclare}{{\sc MP-Declare}}
\newcommand{\Declare}{{\sc Declare}}
%\newcommand{\Submit}{{\sc Submit Loan Application}}
\newcommand{\Assess}{{\sc Assess Application}}
\newcommand{\Notify}{{\sc Notify Outcome}}
\newcommand{\Career}{{\sc Check Career}}
\newcommand{\Medical}{{\sc Check Medical History}}

\begin{document}
	%
	%%%%%%%%%%%%%%%%%%%%%%%%%%%%%%%%%%%%%%%%%%%%%%%%%%%%%%%%%%%%%%%%
	% Used acronyms
	%%%%%%%%%%%%%%%%%%%%%%%%%%%%%%%%%%%%%%%%%%%%%%%%%%%%%%%%%%%%%%%%
	%
	\input{addons/acronyms}
	%
	%%%%%%%%%%%%%%%%%%%%%%%%%%%%%%%%%%%%%%%%%%%%%%%%%%%%%%%%%%%%%%%%
	% Main content
	%%%%%%%%%%%%%%%%%%%%%%%%%%%%%%%%%%%%%%%%%%%%%%%%%%%%%%%%%%%%%%%%
	%
	% Start of the contributions
	\mainmatter
	% In order to avoid that text be beyond margins
	\begin{sloppypar}
		%
		\title{Say It In Your Own Words: Defining Declarative Process Models Using Speech Recognition\thanks{Work supported by the Estonian Research Council (project PRG887).}
		}%
		
		%
		% abbreviated title (for running head) also used for the TOC unless \toctitle is used
		%\titlerunning{Ensuring Model Consistency in Declarative Process Discovery}
		%
		\author{{Han~{van~der~Aa}}\inst{1} %
			\and {{Karl~Johannes}~Balder}\inst{2}
			\and \\{{Fabrizio~Maria}~Maggi}\inst{3} %
			\and {Alexander~Nolte}\inst{2,4}%
		}
		%
		% abbreviated author list (for running head)
		\authorrunning{H.\ {van der Aa}, K.J.\ Balder, F.M.\ Maggi, A.\ Nolte}
		%
		\institute{University of Mannheim, Germany\\
			\email{han@informatik.uni-mannheim.de}
			\and
			University of Tartu, Estonia\\
			\email{Karl.Johannes.Balder@tudeng.ut.ee,alexander.nolte@ut.ee}
			\and
			Free University of Bozen-Bolzano, Italy\\
			\email{maggi@inf.unibz.it}
            \and
			Carnegie Mellon University, Pittsburgh, PA, USA\\
		}


		%
		% typeset the title of the contribution
		\maketitle
		%
		%%%%%%%%%%%%%%%%%%%%%%%%%%%%%%%%%%%%%%%%%%%%%%%%%%%%%%%%%%%%%%%%
		% Abstract
		%%%%%%%%%%%%%%%%%%%%%%%%%%%%%%%%%%%%%%%%%%%%%%%%%%%%%%%%%%%%%%%%
		%
		\begin{abstract}
In recent years declarative, constraint-based approaches have been proposed to model loosely-structured business processes, mediating between support and flexibility. A notable example is the \Declare\ framework, equipped with a graphical declarative language whose semantics can be characterized with several logic-based formalisms. Up to now, the main problem hampering the use of \Declare\ constraints in practice has been the difficulty with modeling constraints either with a formal notation, difficult to understand for users without a background in temporal logics, or with their graphical notation that has been proved to be unintuitive. In this work, we present and evaluate an analysis toolkit that tries to bypass this issue by providing the user with the possibility of modeling \Declare\ constraints using her/his own way of expressing them. The toolkit contains a \Declare\ modeler equipped with a speech recognition mechanism that takes as input a vocal statement from the user and converts it into the closest (set of) \Declare\ constraint(s). The constraints that can be modeled with the tool cover the entire Multi-Perspective extension of \Declare\ (\MPDeclare) that can not only express control-flow properties, but also other perspectives like data and time. Even if we consider \Declare\ as use case, the tool/evaluation presented in this paper can be considered, to the best of our knowledge, the first attempt to test the usability of speech recognition in business process modeling.
\end{abstract} 
		%
		%%%%%%%%%%%%%%%%%%%%%%%%%%%%%%%%%%%%%%%%%%%%%%%%%%%%%%%%%%%%%%%%
		% Main content
		%%%%%%%%%%%%%%%%%%%%%%%%%%%%%%%%%%%%%%%%%%%%%%%%%%%%%%%%%%%%%%%%
		%
		% !TeX root = ../Main.tex
% !TeX spellcheck = en_US

\section{Introduction}
\label{sec:introduction}

Process models are an important means to capture information on organizational processes, serving as a basis for communication and often as the starting point for analysis and improvement~\cite{dumas2013fundamentals}.
For processes that are relatively structured, \emph{imperative} process modeling notations, such as the Business Process Model and Notation (BPMN), are most commonly employed. However, other processes, in particular knowledge-intensive ones, are more flexible and, therefore, less structured. An important characteristic of such processes is that it is typically infeasible to specify the entire spectrum of allowed execution orders in advance~\cite{dicicio2015knowledge}, which severely limits the applicability of the imperative process modeling paradigm. Instead, such processes are better captured using \emph{declarative} process models defined in process modeling languages like \Declare\ whose semantics
can be characterized using temporal logic properties. These models, indeed, do not require an explicit definition of  all allowed execution orders, but rather use constraints to define the boundaries of the permissible process behavior \cite{vanderaalst2009declarative}.

Although their benefits are apparent, establishing declarative process models is known to be difficult, especially for domain experts that generally lack expertise in temporal logics, and in most of the cases find the graphical notation of \Declare\ constraints unintuitive.
Due to these barriers to declarative model creation, a preliminary approach has been presented in~\cite{vanderaa2019extracting} that automatically extracts declarative process models from natural language texts, similarly to what many works focusing on the generation of imperative process models from natural language descriptions have largely investigated (cf., \cite{friedrich2011process,schumacher2013extracting,vanderaa2018checking,de2017assisting}).
%processes are often documented using natural language instead~\cite{}.  Recognizing this, several works have been developed that automatically extract process models from natural language texts (cf., \cite{friedrich2011process,schumacher2013extracting,vanderaa2018checking,de2017assisting}).
%The vast majority of these focus on the generation of imperative process descriptions, whereas only one preliminary approach~\cite{vanderaa2019extracting} targets the extraction of declarative process models.

%In this work, we present and evaluate a tool that tries to bypass this issue by providing the user with the possibility of modeling \Declare\ constraints using his/her own way of expressing them. The tool is a \Declare\ modeler equipped with a vocal assistant that takes as input a vocal statement from the user and converts it into the closest (set of) \Declare\ constraint(s). The inputs that can be provided to the tool cover the entire Multi-Perspective extension of \Declare\ (MP-\Declare\) that can not only express control-flow properties, but also other perspectives like data and time. Even if we consider \Declare\ as use case, the tool/evaluation presented in this paper can be considered, to the best of our knowledge, the first attempt to test the usability of speech recognition in business process modeling.

%we aim to support the elicitation of declarative process models based on natural language input. In particular,

In this work, we go beyond the extraction of \Declare\ constraints from natural language descriptions, and present and evaluate an interactive approach that takes as input a vocal statement from the user and employs speech recognition to convert it into the closest (set of) \Declare\ constraint(s). The approach has been integrated in a declarative modeling and analysis toolkit. With this tool, the user is not required to have any experience in temporal logics nor to be familiar with the graphical notation of \Declare\ constraints, but can express temporal properties using his/her own words. The temporal properties that can be modeled with the tool cover the entire Multi-Perspective Declare (\MPDeclare) language that can not only express control-flow properties, but also other perspectives like data, time and resource. An important contribution of the paper is that the evaluation presented in this paper can be considered, to the best of our knowledge, the first attempt to test the usability of speech recognition in business process modeling.

\todoinline{describe evaluation results}

To summarize, we are able to provide the following contributions in comparison to the state of the art:
\begin{compactenum}
	\item we greatly increase the coverage of \Declare\ constraint types wrt.~\cite{vanderaa2019extracting};
	\item we handle more flexible inputs so that more expressions can be recognized by the speech recognition component;
    \item constraints are now linked to each other, enabling the definition of actual declarative process models, rather than just individual constraints;	
    \item no one-shot approach;  \todo{what does this mean?}
	\item we ensure additional expressiveness by incorporating conditions from data, time, and resource perspectives
	\item the generated process models can be directly used as input for, e.g., conformance checking, log generation, and process monitoring (being the modeler integrated in an analysis toolkit);
    \item we test the usability of speech recognition in business process modeling.
\end{compactenum}


%- finally, all of the above is contained in a tool that uses speech recognition for the efficient elicitation of constraints, employs interaction to resolve ambiguity and enable step-by-step constraint enrichment, both its intermediatry and final output are directly available as part of a state-of-the-art declarative modeling toolkit (RuM). what can the user then do with these models because of this?


\todoinline{describe remainder of paper}






		% !TeX root = ../main.tex

\section{Background}
\label{sec:background}
%\todoinline{We have to decide which constraints we want to show here. I think that the constraints shown in this section are not the same as the ones mentioned later.}

In this section, we first introduce \Declare\ and \MPDeclare\ and then discuss the main challenges associated with the translation of  natural language into declarative constraints.

\subsection{Declarative Process Modeling}
\label{sec:bg:declarativemodeling}
\Declare\ is a declarative process modeling language originally introduced by
Pesic and van der Aalst in \cite{Pesic2007:DECLARE}. Instead of explicitly
specifying the flow of the interactions among process activities, \Declare\
describes a set of constraints that must be satisfied throughout the process
execution. The possible orderings of activities are implicitly specified by
constraints and anything that does not violate them is possible during execution.
\MPDeclare\ is the Multi-Perspective extension of \Declare\ that was first introduced in \cite{Burattin2015} and can express constraints over perspectives of a process like data, time and resources.

To explain the semantics of \Declare\ and \MPDeclare, we have to introduce some preliminary notions. In particular, we call a \emph{case} an ordered sequence of events representing a single ``run'' of a process (often referred to as a \emph{trace} of events). Each event in a trace refers to an \emph{activity} (i.e., a well-defined step in a business process), has a \emph{timestamp} indicating when the event occurs and can have additional \emph{data attributes} collected into a \emph{payload}. Consider, for example, the occurrence of an event \emph{ship order} (O) and suppose that, after the occurrence of O at timestamp $\tau_{O}$, the attributes \emph{customer type} and \emph{amount} have values \emph{gold} and $155\euro$. In this case, we say that, when O occurs, two special relations are valid \emph{event}(O) and $p_{O}($\emph{gold},$155\euro)$. In the following, we identify \emph{event}(O) with the event itself O and we call (\emph{gold},155), the \emph{payload} of O.
%In this paper, we assume that all attributes are globally visible and can be accessed/manipulated by all activity instances executed inside the case.


\begin{table}[tb]
	\caption{Semantics for Declare templates \label{tbl:timed-ltl}}
	\centering
	\scriptsize{
		\begin{tabular}{llc}
			\toprule
			\textbf{Template}    & \textbf{LTL semantics} &  \textbf{Activation}\\
			%\midrule
			%existence & $\lfuture A $ & $A$\\
			%absence & $\neg \lfuture A$ & $A$ \\
			%exclusive existence & $-$ & $\lfuture A \wedge \neg \ensuremath{\mathbf{F}_{[0,a]}}A \wedge \neg \ensuremath{\mathbf{F}_{[b,\infty]}} A$ \\
			%exclusive allowance & $\neg \ensuremath{\mathbf{F}_{[0,a]}}A \wedge \neg \ensuremath{\mathbf{F}_{[b,\infty]}} A$ \\
			%exactly(1) & $\lfuture A \wedge \neg \lfuture (A \wedge \lnext (\lfuture A)) $ & $\lfuture A \wedge \neg \lfuture (A \wedge \lnext (\lfuture A)) $ \\
			%|rule
			\midrule
			responded existence  & $\lglobally(A \rightarrow (\lonce B \vee \lfuture B))$ & $A$\\
			%co-existence & $\lfuture A \leftrightarrow \lfuture A$ & $\lglobally(A \leftrightarrow (\lonce B \vee \lfuture B))$ \\
			\midrule
			response &  $\lglobally(A \rightarrow \lfuture B)$ & $A$ \\
			alternate response  & $ \lglobally(A \rightarrow \lnext(\neg A \luntil B))$ & $A$\\
			%chain response & $\lglobally(A \rightarrow \lnext B)$ & $\lglobally(A \rightarrow (\lnext B \wedge (\neg B \luntil B)))$ \\
			chain response &  $\lglobally(A \rightarrow \lnext B)$& $A$ \\
			\midrule
			%precedence & $\neg B~ \lwuntil A$ & $(\neg B ~\lwuntil A) ~ \wedge ~ \lglobally(A \rightarrow ( (\neg B~ \lwuntil A) \vee \lfuture B)$ \\
			precedence &  $\lglobally(B \rightarrow \lonce A)$ & $B$\\
			%alternate precedence & $(\neg B\luntil A) \vee ( B \rightarrow \lnext(\neg B \luntil A))$ & $(\neg B ~\lwuntil A) ~ \wedge ~ \lglobally(A \rightarrow ( \lglobally(\neg B) \vee (\neg A \luntil B))$ \\
			alternate precedence & $\lglobally(B \rightarrow \lyesterday(\neg B \lsince A ))$ & $B$\\
			%chain precedence & $\lglobally(\lnext B \rightarrow A)$ & $\lglobally(A \rightarrow ( (\lnext \neg B) \vee (\lnext B \wedge (\neg B \luntil B)))$ \\
			chain precedence & $\lglobally(B  \rightarrow \lyesterday A)$ & $B$\\
			\midrule
			%succession & $\lglobally(B \rightarrow \lonce A) ~ \wedge \lglobally(A \rightarrow \lfuture B)$ & $\lglobally(B \rightarrow \lonce A) ~ \wedge ~ \lglobally(A \rightarrow \lfuture B) $ \\
			%\multirow{2}{*}{alternate succession} & $\lglobally(B \rightarrow \lyesterday(\neg B \lsince A)) ~ \wedge$ & $\lglobally(B \rightarrow \lyesterday(\neg B \lsince A)) ~ %\wedge$ \\
			% & $  \lglobally(A \rightarrow \lnext(\neg A \luntil B))$ & $\lglobally(A \rightarrow \lnext(\neg A \luntil B))$ \\
			%chain succession & $\lglobally(B \rightarrow \lyesterday A) ~ \wedge \lglobally(A \rightarrow \lnext B)$ & $\lglobally(B \rightarrow \lyesterday A) ~ \wedge \lglobally(A \rightarrow \lnext B)$ \\
			% |rule
			not responded existence  &
			$\lglobally(A \rightarrow \neg (\lonce B \vee \lfuture B ))$ & $A$\\
			not response  & $\lglobally(A \rightarrow \neg \lfuture B )$ & $A$\\
			not precedence & $\lglobally(B \rightarrow \neg \lonce A )$ & $B$\\
			%not succession & $\lglobally(A \rightarrow \neg (\lfuture B))$ & $\lglobally(A \rightarrow \neg (\lfuture B))$ \\
			%not chain succession & $\lglobally(A \rightarrow \neg (\lnext B))$ & $\lglobally(A \rightarrow \neg (\lnext B \wedge (\neg B \luntil B)))$ \\
			not chain response  & $\lglobally(A \rightarrow \neg \lnext B )$ & $A$\\
			not chain precedence  & $\lglobally(B \rightarrow \neg \lyesterday A )$ & $B$\\
			%not chain succession & $\lglobally(A \rightarrow \neg (\lnext B))$ & $\lglobally(A \rightarrow \neg (\lnext B))$ \\
			\bottomrule
		\end{tabular}
	}
\end{table}

\subsubsection{Declare}
\label{sec:declare}

%In
%comparison with imperative approaches that produce ``closed'' models, i.e., all
%that is not explicitly specified is forbidden, \Declare\ models are ``open'' and
%tend to offer more possibilities for the execution. In this way, \Declare\ enjoys
%flexibility and is very suitable
%for highly dynamic processes characterized by high complexity and variability
%due to the changeability of their execution environments.

A \Declare\ model consists of a set of constraints applied to
activities. Constraints, in turn, are based on templates. Templates
are patterns that define parameterized classes of properties, and
constraints are their concrete instantiations  (we indicate template parameters with capital letters and concrete activities in their instantiations with lower case letters).
Templates have a graphical representation and
their semantics can be formalized using different logics
\cite{Montali2010:Choreographies}, the main one being LTL over finite
traces, making them verifiable and executable.  Each constraint
inherits the graphical representation and semantics from its
template.
\todo{We don't consider existence constraints}
%The major benefit of using templates is that analysts do
% not have to be aware of the underlying logic-based formalization to
% understand the models. They work with the graphical representation of
% templates, while the underlying formulas remain
% hidden.
\tablename~\ref{tbl:timed-ltl} summarizes some Declare templates (the
reader can refer to \cite{declareCSRD09} for a full description of the
language).
Here, the $\lfuture$, $\lnext$, $\lglobally$, and $\luntil$ LTL (future)
operators have the following intuitive meaning: formula $\lfuture \phi_1$ means
that $\phi_1$ holds sometime in the future, $\lnext \phi_1$ means that $\phi_1$
holds in the next position, $\lglobally \phi_1$ says that $\phi_1$ holds forever in the future, and, lastly,
$\phi_1 \luntil \phi_2$ means that sometime in the future $\phi_2$ will hold and
until that moment $\phi_1$ holds (with $\phi_1$ and $\phi_2$ LTL formulas).  The $\lonce$, $\lyesterday$, and $\lsince$ LTL (past)
operators have the following meaning: $\lonce \phi_1$ means
that $\phi_1$ holds sometime in the past, $\lyesterday \phi_1$ means that $\phi_1$
holds in the previous position, and, lastly, $\phi_1 \lsince \phi_2$ means that sometime in the past $\phi_2$
holds and since that moment $\phi_1$ holds.

Consider, for example, the \emph{response} constraint $\lglobally(a \rightarrow \lfuture b)$. This constraint indicates that if $a$ {\it occurs}, $b$ must
eventually {\it follow}.
Therefore, this constraint is satisfied for traces such as $\textbf{t}_1$ =
$\langle a, a, b, c \rangle$, $\textbf{t}_2 = \langle b,
b, c, d \rangle$, and $\textbf{t}_3 = \langle a, b,
c, b \rangle$, but not for $\textbf{t}_4 = \langle a, b,
a, c \rangle$ because, in this case, the second instance of $a$ is not followed by a $b$. Note that, in $\textbf{t}_2$,
the considered response constraint is satisfied in a trivial way because $a$ never occurs.
%In this case, we say that the constraint is \emph{vacuously
%	satisfied}~\cite{kupf:vacu03}. In \cite{Burattin2012}, the authors introduce
%the notion of \emph{behavioral vacuity detection} according to which a
%constraint is non-vacuously satisfied in a trace when it is activated in that
%trace. 
An \emph{activation} of a constraint in a trace is an event whose
occurrence imposes, because of that constraint, some obligations on other events (targets)
in the same trace. For example, $a$ is an activation for the \emph{response}
constraint $\lglobally(a \rightarrow \lfuture b)$ and $b$ is a target, because the execution of $a$ forces $b$ to be executed, eventually. In \tablename~\ref{tbl:timed-ltl}, for each template the corresponding activation is specified.



\begin{table*}[t!]
	\caption{Semantics for \MPDeclare\ constraints \label{tbl:timed-mfotl}}
	\centering
	\scriptsize{
		\begin{tabular}{ll}
			\toprule
			\textbf{Template} & \textbf{MFOTL Semantics} \\
			%\midrule
			%existence & $\tfuture (A \wedge \exists x.\varphi_a(x)) $ \\
			%absence & $\neg \tfuture (A \wedge \exists x.\varphi_a(x))$ \\
			%exclusive existence & $-$ & $\tfuture A \wedge \neg \ensuremath{\mathbf{F}_{[0,a]}}A \wedge \neg \ensuremath{\mathbf{F}_{[b,\infty]}} A$ \\
			%exclusive allowance & $\neg \ensuremath{\mathbf{F}_{[0,a]}}A \wedge \neg \ensuremath{\mathbf{F}_{[b,\infty]}} A$ \\
			%exactly(1) & $\lfuture A \wedge \neg \lfuture (A \wedge \lnext (\lfuture A)) $ & $\tfuture A \wedge \neg \lfuture (A \wedge \lnext (\lfuture A)) $ \\
			%\midrule
			\midrule
			responded existence  & $\lglobally( \forall x.((A \wedge \varphi_a(x)) \rightarrow (\tonce (B   \wedge \exists y.\varphi_c(x,y)) \vee \tfuture (B \wedge \exists y.\varphi_c(x,y)))))$ \\
			%co-existence & $\lfuture A \leftrightarrow \lfuture A$ & $\lglobally(A \leftrightarrow (\tonce B \vee \tfuture B))$ \\
			\midrule
			response &  $\lglobally( \forall x. ((A \wedge \varphi_a(x)) \rightarrow \tfuture (B \wedge \exists y.\varphi_c(x,y))))$ \\
			alternate response  & $ \lglobally(\forall x. ((A \wedge \varphi_a(x)) \rightarrow \lnext(\neg (A \wedge \varphi_a(x)) \tuntil (B \wedge \exists y.\varphi_c(x,y)))))$ \\
			%chain response & $\lglobally(A \rightarrow \lnext B)$ & $\lglobally(A \rightarrow (\lnext B \wedge (\neg B \tuntil B)))$ \\
			chain response &  $\lglobally(\forall x. ((A \wedge \varphi_a(x)) \rightarrow \tnext (B \wedge \exists y.\varphi_c(x,y)))$ \\
			\midrule
			%precedence & $\neg B~ \lwuntil A$ & $(\neg B ~\lwuntil A) ~ \wedge ~ \lglobally(A \rightarrow ( (\neg B~ \lwuntil A) \vee \tfuture B)$ \\
			precedence &  $\lglobally(\forall x. ((B \wedge \varphi_a(x)) \rightarrow \tonce (A \wedge \exists y.\varphi_c(x,y)))$ \\
			%alternate precedence & $(\neg B\luntil A) \vee ( B \rightarrow \lnext(\neg B \luntil A))$ & $(\neg B ~\lwuntil A) ~ \wedge ~ \lglobally(A \rightarrow ( \lglobally(\neg B) \vee (\neg A \tuntil B))$ \\
			alternate precedence & $ \lglobally(\forall x. ((B \wedge \varphi_a(x)) \rightarrow \lyesterday(\neg (B \wedge \varphi_a(x)) \tsince (A \wedge \exists y.\varphi_c(x,y))))$ \\
			%chain precedence & $\lglobally(\lnext B \rightarrow A)$ & $\lglobally(A \rightarrow ( (\lnext \neg B) \vee (\lnext B \wedge (\neg B \tuntil B)))$ \\
			chain precedence & $\lglobally(\forall x. ((B \wedge \varphi_a(x)) \rightarrow \tyesterday (A \wedge \exists y.\varphi_c(x,y)))$ \\
			\midrule
			%succession & $\lglobally(B \rightarrow \lonce A) ~ \wedge \lglobally(A \rightarrow \lfuture B)$ & $\lglobally(B \rightarrow \tonce A) ~ \wedge ~ \lglobally(A \rightarrow \tfuture B) $ \\
			%\multirow{2}{*}{alternate succession} & $\lglobally(B \rightarrow \lyesterday(\neg B \lsince A)) ~ \wedge$ & $\lglobally(B \rightarrow \lyesterday(\neg B \tsince A)) ~ %\wedge$ \\
			% & $  \lglobally(A \rightarrow \lnext(\neg A \luntil B))$ & $\lglobally(A \rightarrow \lnext(\neg A \tuntil B))$ \\
			%chain succession & $\lglobally(B \rightarrow \lyesterday A) ~ \wedge \lglobally(A \rightarrow \lnext B)$ & $\lglobally(B \rightarrow \tyesterday A) ~ \wedge \lglobally(A \rightarrow \tnext B)$ \\
			% \midrule
			not responded existence  &
			$\lglobally( \forall x.((A \wedge \varphi_a(x)) \rightarrow \neg (\tonce (B   \wedge \exists y.\varphi_c(x,y)) \vee \tfuture (B \wedge \exists y.\varphi_c(x,y)))))$ \\
			not response  & $\lglobally( \forall x. ((A \wedge \varphi_a(x)) \rightarrow \neg \tfuture (B \wedge \exists y.\varphi_c(x,y))))$ \\
			not precedence & $\lglobally(\forall x. ((B \wedge \varphi_a(x)) \rightarrow \neg \tonce (A \wedge \exists y.\varphi_c(x,y)))$ \\
			%not succession & $\lglobally(A \rightarrow \neg (\lfuture B))$ & $\lglobally(A \rightarrow \neg (\tfuture B))$ \\
			%not chain succession & $\lglobally(A \rightarrow \neg (\lnext B))$ & $\lglobally(A \rightarrow \neg (\lnext B \wedge (\neg B \tuntil B)))$ \\
			not chain response  & $\lglobally(\forall x. ((A \wedge \varphi_a(x)) \rightarrow \neg \tnext (B \wedge \exists y.\varphi_c(x,y)))$ \\
			not chain precedence  & $\lglobally(\forall x. ((B \wedge \varphi_a(x)) \rightarrow \neg \tyesterday (A \wedge \exists y.\varphi_c(x,y)))$ \\
			%not chain succession & $\lglobally(A \rightarrow \neg (\lnext B))$ & $\lglobally(A \rightarrow \neg (\tnext B))$ \\
			\bottomrule
	\end{tabular}}
\end{table*}




\subsubsection{Multi-Perspective Declare}
\label{sec:semantics}
\MPDeclare extends \Declare\ with additional perspectives. 
Its semantics is expressed in Metric First-Order Linear Temporal Logic (MFOTL) and is shown in Table~\ref{tbl:timed-mfotl}. We describe here the semantics informally and we refer the interested reader to \cite{Burattin2015} for more details.\todo{Han: If we say anyway that we describe semantics informally, can't we drop Table 2?}
%While many reasoning
%tasks are clearly undecidable for MFOTL, this logic is appropriate to unambiguously
%describe the semantics of the Multi-Perspective Declare~constraints we can use for conformance checking in our proposed algorithms.
%To explain the semantics, we have to introduce some preliminary notions.



%To define the new semantics for Declare, we have to contextualize the definitions given in Section \ref{sec:mfotl} in XES. Consider, for example, that the execution of an activity $pay$ is recorded in an event log and, after the execution of $pay$ at timestamp $\tau_i$, the attributes $originator$, $amount$, and $z$ have values $John$, $100$, and $July$. In this case, the valuation of $(activityName,originator,amount,z)$ is $(pay,John,100,July)$ in $\tau_i$. Considering that in XES, by definition, the activity name is a special attribute always available, if $(pay,John,100,July)$ is the valuation of $(activityName,originator,amount,z)$, we say that, when $pay$ occurs, two special relations are valid $event(pay)$ and $p_{pay}(John,100,July)$. In the following, we identify $event(pay)$ with the event itself $pay$ and we call $(John,100,July)$, the \emph{payload} of $pay$.
%\emph{ship order}

%Note that all the templates in \MPDeclare\ in Table~\ref{tbl:timed-mfotl} have two parameters, an activation and a target (see also \tablename~\ref{tbl:timed-ltl}).
The standard semantics of \Declare\ is extended by requiring additional conditions on data, i.e., the \emph{activation condition}, the \emph{correlation condition}, and a \emph{time condition.} As an example, we consider the response constraint ``\emph{ship order} is always eventually followed by \emph{send invoice}'' having \emph{ship order} as activation and \emph{send invoice}\ as target.
The activation condition $\varphi_a$ is a relation that must be valid when the activation occurs. If the activation condition does not hold the constraint is not activated. The activation condition has the form $p_A(x) \wedge r_{a}(x)$, meaning that when $A$ occurs with payload $x$, the relation $r_a$ over $x$ must hold. For example, we can say that whenever \emph{ship order} occurs, the order amount is higher than $\euro{100}$ euros, and the customer is of type \emph{gold}, eventually an invoice must be sent. In case \emph{ship order} occurs but these constraints are not met, the constraint is not activated.
%the amount is lower than $\euro{100}$ euros or the customer type is different from \emph{gold}, the constraint is not activated.


The correlation condition  $\varphi_c$ is a relation that must be valid when the target occurs. It has the form $p_B(y) \wedge r_{c}(x,y)$, meaning that when $B$ occurs with payload $y$, the relation $r_c$ involving the payload $x$ of $A$ and the payload $y$ of $B$ must hold. A special type of correlation condition has the form $p_B(y) \wedge r_{c}(y)$, which we call \emph{target condition}, since it does not involve the payload of the activation.

%In this paper, we aim at discovering constraints that correlate an activation and a target condition. For example, we can find that whenever \emph{ship order} occurs, and the amount of the loan is higher than $50\,000$ euros and the applicant has a salary lower than $24\,000$ euros per year, then eventually \emph{send invoice}\ must follow, and the assessment type will be $Complex$ and the cost of the assessment higher than $100$ euros.

Finally, in \MPDeclare, also a time condition can be specified through an interval ($I=[\tau_0,\tau_1)$) indicating the minimum and the maximum temporal distance allowed between the occurrence of the activation and the occurrence of the corresponding target.
%In the following, we explain in detail the intuitive meaning of some of the constraints in Table~\ref{tbl:timed-mfotl}. We indicate with $\tau_A$ and $\tau_B$ the timestamps of $A$ and $B$, respectively.
%The \emph{responded existence} constraint, in Table~\ref{tbl:timed-mfotl}, indicates that, if $A$ occurs at time $\tau_A$ with $\varphi_a$ holding true, $B$ must occur at some point $\tau_B\in[\tau_A + \tau_0, \tau_A + \tau_1)$ with $\varphi_c$ holding true.
%The response constraint indicates that, if $A$ occurs at time $\tau_A$ with $\varphi_a$ holding true, $B$ must occur at some point $\tau_B\in[\tau_A + \tau_0, \tau_A + \tau_1)$ with $\varphi_c$ holding true. The alternate response constraint specifies that,
%if $A$ occurs at time $\tau_A$ with $\varphi_a$ holding true, $B$ must occur at some point $\tau_B\in[\tau_A + \tau_0, \tau_A + \tau_1)$ with $\varphi_c$ holding true. $A$ is not allowed in the interval $[\tau_A,\tau_B]$ if $\varphi_a$ is true. Any event different from $A$ is allowed and, also, $A$ is allowed if $\varphi_a$ is false.
%The chain response constraint indicates that, if $A$ occurs at time $\tau_A$ with $\varphi_a$ holding true, $B$ must occur next at some point $\tau_B\in[\tau_A + \tau_0, \tau_A + \tau_1)$ with $\varphi_c$ holding true.


%\todoinline{we could move this subsection to the following section?}

\subsection{From Natural Language to Declarative Models}
\label{sec:bg:nltodeclarative}

A crucial component of our work involves the extraction of declarative constraints from natural language. This extraction step involves the identification of the described actions (activities), as well as the identification of the constraint that applies to these actions.
Due to the inherent flexibility of natural language, this extraction step can be highly challenging. Its difficulty manifests itself in the sense that, on the one hand, the same declarative constraint can be expressed in a wide variety of manners, whereas, on the other hand, subtle textual differences can completely change the meaning of the described constraint. 
%These two complimentary challenges can be illustrated as follows:

\begin{table}
	\begin{tabular}{cll}
		\toprule
		\textbf{ID} & \textbf{Description} \\
		\midrule
		$s_1$ & An invoice must be created before the invoice can be approved. \\
		$s_2$ & A bill shall be created prior to it being approved. \\
		$s_3$ & Invoice creation must precede its approval. \\
		$s_4$ & Approval of an invoice must be preceded by its creation. \\
		$s_5$ & Before an invoice is approved, it must be created. \\
		\bottomrule
	\end{tabular}
	\caption{Different descriptions of \textit{Precedence(create order, approve order)}}
	\label{tab:challenge1}			
\end{table}

\mypar{Variability of textual descriptions.} As shown in \autoref{tab:challenge1}, the same declarative constraint can be described in a broad range of manners. Key types of differences occur due to:
the use of synonyms (e.g., \emph{create invoice} in $s_1$  and \emph{create bill} in $s_2$) and due to different grammatical structures (e.g., $s_1$ uses verbs to denote activities, whereas $s_3$ uses nouns, like ``\emph{invoice creation}'').
Finally, constraint descriptions can differ in the order in which they describe the different components of binary constraints, i.e., whether they describe the constraint in a chronological fashion, e.g., $s_1$ to $s_3$, or the reverse order, such as $s_4$ and $s_5$.

To achieve our goal of supporting users in the elicitation of declarative process models, we must not limit the user too much in terms of the input that they can provide. Rather, it has to be accommodating to the different manners in which users may choose to describe constraints. However, a successful approach must be able to do this while simultaneously being able to recognize subtle distinctions between different constraints, as discussed next.


%\begin{compactitem}
%	\item \textbf{Synonyms.} Synonyms are omnipresent in any unstructured or semi-structured natural language text. Their presence impacts two aspects of constraint descriptions. First, synonymous terms or phrases can be used to refer to what is semantically the same action, e.g., \emph{create invoice} in $s_1$ and \emph{create bill} in $s_2$. Second, they can be used to describe the inter-relations between actions in different manners. E.g., ``\emph{before the invoice can be approved}'' in $s_1$ has the same implications for the declarative constraint as ``\emph{prior to it being approved}'' in $s_2$.
%	
%	\item \textbf{Different grammatical structures.} Textual descriptions of constraints can have widely different grammatical structures. Given that such a description captures a number of actions and their inter-relations, a particularly important distinction is to be made between descriptions that use \emph{verbs} to denote actions, e.g., ``\emph{An invoice must be created}'' in $s_1$, versus those that use \emph{nouns} to denote actions, e.g., ``\emph{invoice creation}'' in $s_3$.
%	%		The latter cases are more difficult to handle, given that most work on the extraction of process information from text only uses verb-based identification of actions~\cite{vanderaa2019extracting}.
%	%		
%	\item \textbf{Description order.} Finally, constraint descriptions can differ in the order in which they describe the different components of binary constraints, i.e., whether they describe the constraint in a chronological fashion, e.g., $s_1$ to $s_3$, or the reverse order, such as $s_4$ and $s_5$.
%	
	
	
	
	
%\end{compactitem}

%\noindent  To achieve our goal of supporting users in the elicitation of declarative process models, we must not limit the user too much in terms of the input that they can provide. Rather, it has to be accommodating to the different manners in which users may choose to describe constraints. However, a successful approach must be able to do this while simultaneously being able to recognize subtle distinctions between different constraints, as discussed next.

\mypar{Subtle differences leading to different constraints.}
Small textual differences can have a considerable impact on the semantics of a constraint description and, thus, on the constraints that should be extracted from them. To illustrate this, consider the descriptions in \autoref{tab:challenge2}. In comparison to description $s_6$, the three other descriptions each differ by only a single word. However, as shown in the right-hand column, the described constraints vary greatly. 
For instance, 
the difference between the Response constraint of $s_6$ and the Precedence constraints described by $s_7$ lies in the obligation associated with the \emph{send invoice} action. The former specifies that this \emph{must} occur, indicating an obligation, whereas the latter specifies that it \emph{can} occur. 
Further, the direction in which a constraint is described is often signaled through small textual,  typically through the  use of temporal prepositions. For instance, in $s_8$, the use of \emph{first} completely reverses the meaning of the described constraint.
 Finally, it should be clear that the presence of negation drastically changes the meaning of a constraint, as seen for $s_9$. The addition of \emph{not} to description $s_6$ changes Response into a Not Succession.

\begin{table}
	\caption{Subtle textual differences ($A$ as \emph{ship order}, $B$ as \emph{send invoice})}
	\label{tab:challenge2}
	\begin{tabular}{cll}
		\begin{tabular}{cll}
			\toprule
			\textbf{ID} & \textbf{Description}  & \textbf{Constraint}\\
			\midrule
			$s_6$ & If an order is shipped, an invoice must be sent. & Response($A$, $B$)\\
			$s_7$ & If an order is shipped, an invoice can be sent. & Precedence($A$, $B$)\\
			$s_8$ & If an order is shipped, an invoice must be sent first. & Precedence($B$, $A$)\\
			$s_9$ & If an order is shipped, an invoice must not be sent.
			& NotSuccn.($A$, $B$) \\
			\bottomrule
		\end{tabular}
	\end{tabular}
\end{table}

%
%\begin{compactitem}
%	\item \textbf{Obligation.} The difference between the Response constraint of $s_6$ and the Precedence constraints described by $s_7$ lies in the obligation associated with the \emph{send invoice} action. The former specifies that this action \emph{must} occur, indicating an obligation, whereas the latter specifies that it \emph{can} occur. As such, the activation and the target of the constraint are swapped.
%	
%	
%	\item \textbf{Order indicators.} As discussed before, constraints can be described in chronological and non-chronological orders. This difference is often signaled through small textual clues,  typically through the  use of temporal prepositions. This is seen for description $s_8$, where the use of \emph{first} completely reverses the meaning of the described constraint.
%	
%	\item \textbf{Negation.} Finally, it should be clear that the presence of negation drastically changes the meaning of a constraint, as seen for $s_9$. The addition of \emph{not} to description $s_6$ changes the original Response constraint into a Not Succession. \todoinline{what to say about NotSuccession? I have added one sentence. Is it ok?}
%	
%\end{compactitem}



\mypar{State of the art}
So far, only one approach has been developed to automatically extract declarative process constraints from natural language text. This approach, by Van der Aa et al.~\cite{vanderaa2019extracting}, is able to extract five types of Declare templates, Init, End, Precedence, Response, Succession, as well as  their negated forms. 
%The approach aims at handling the aforementioned extraction challenges, in particular based on a unique approach to the extraction of noun-based actions. \todo{not very clear} 
Its evaluation results show that it is able to handle a reasonable variety of inputs.
Recognizing its potential, we build on this approach. In particular, we use this technique for activity extraction and employ it for the extraction of the aforementioned five constraint templates.

Nevertheless, due to its limitations, we extend this approach in two main manners: (1) we generalized some of the pattern recognition mechanisms in order to handle more flexible inputs, (2) we cover eight additional constraint templates, and (3) we add support for the identification of multi-perspective constraints.





		% !TeX root = ../Main.tex
\section{The approach}
\label{sec:framework}
		\section{The Tool}
\label{sec:implementation}
The proposed approach was implemented in RuM, a modeling and analysis toolkit for Rule Mining.\footnote{The tool can be found at \url{https://tinyurl.com/r3zm45f}} The approach presented in this paper is the modeling component of the tool called Speech2RuM. In \figurename~\ref{fig:toolNew}, a screenshot of the latest version of the tool is shown. In the top-left area of the screen the recognized vocal input is shown. It is possible to record a constraint (control-flow perspective) as shown in \figurename~\ref{fig:controlflow} where an Init and a Response constraint are modeled. After having inserted a standard \Declare\ constraint, the user can record an activation, a correlation and a time condition as shown in \figurename~\ref{fig:toolNew} where an activation (\emph{amount} is greater than 100 and \emph{customer type} is \emph{gold}) and a correlation (activation and target share the same value for attribute \emph{amount}) condition have already been recorded and a time condition is extracted from the vocal input \textit{``not before 2 hours and within 12 hours''}. The constraints expressed with their graphical notation are shown in the top-right area of the screen, and modifiable lists of activities and constraints are shown in the bottom-left and bottom-right areas of the screen.

\begin{figure}[t!]
	\includegraphics[width=\textwidth]{figures/toolAfter}
	\caption{Screenshot of the tool}
	\label{fig:toolNew}
\end{figure}


\begin{figure}[t!]
  \centering
  \subfloat[Inserting an Init constraint \label{fig:init}]{
    \includegraphics[scale = 0.4]{figures/init}
  }\hspace{0.2cm}
  \subfloat[Inserting a Response constraint \label{fig:response}]{
    \includegraphics[scale = 0.4]{figures/response}
  }
\caption{Inserting control-flow properties.}
\label{fig:controlflow}
\end{figure}


		% !TeX root = ../Main.tex
\section{User Evaluation}
\label{sec:experiments}
		% !TeX root = ../Main.tex
\section{Related work}
\label{sec:relatedwork}

Our research relates to three streams of research: Consistency checking for knowledge bases, research on process mining, and specifically research on \gls{declare}.
Research in the area of knowledge representation has considered the issue of consistency checking. In particular, in the context of Knowledge-based configuration systems, Felfernig et al.~\cite{DBLP:journals/ai/FelfernigFJS04} have challenged the problem of finding the core cause of inconsistencies within the knowledge base during its update test, in terms of minimal conflict sets (the so-called diagnosis). The proposed solution relies on the recursive partitioning of the (extended) CSP problem into subproblems, skipping those that do not contain an element of the propagation-specific conflict~\cite{DBLP:conf/aaai/Junker04}. In the same research context, the work described in \cite{felfernig2011corediag} focuses on the detection of non-redundant constraint sets. The approach is again based on a divide-and-conquer approach, that favours however those constraints that are ranked higher in a lexicographical order. Differently from such works, we tend to exploit the characteristics of \gls{declare} \glspl{contemp} in a sequential exploration of possible solutions. As in their proposed solutions, though, we base upon a preference-oriented ranking when deciding which \glspl{con} to keep in the returned set.


The problem of consistency arises in process mining when working with behavioural constraints. Constraint sets as those of the $\alpha$ algorithm~\cite{Aalst.etal/2004:WorkflowMiningDiscovering} and its extension~\cite{Wen.etal/DMKD2007:AlphaPlusPlus} or behavioural profiles~\cite{DBLP:journals/tse/WeidlichMW11,DBLP:journals/is/WeidlichPDMW11} are per construction consistent. DCR graphs are not directly discussed from the perspective of consistency~\cite{DBLP:conf/bpm/ReijersSS13}, but benefit from our work due to their grounding in B{\"u}chi automata.

More specifically, our work is related to research on \gls{declare} and strategies to keep sets small and consistent.
In \cite{Maggi.etal/CIDM2011:UserGuidedDiscovery}, the authors present an approach based on the instantiation of a set of candidate \gls{declare} constraints that are checked with respect to the log to identify the ones that are satisfied in a higher percentage of traces. This approach has been improved in \cite{Maggi.etal/CAiSE2012:EfficientDiscoveryUnderstandable} by reducing the number of candidates to be checked through an apriori algorithm.
In \cite{Maggi.etal/CAiSE2013:KnowledgeBasedIntegrated}, the same approach has been applied for the repair of \gls{declare} models based on log and for guiding the discovery task based on apriori knowledge provided in different forms. In this work, some simple reduction rules are presented. These reduction rules are, however, not sufficient to detect redundancies due to complex interactions among constraints in a discovered model as demonstrated in our experimentation.
%In \cite{MCMM13}, an approach is presented to discover Declare constraints based on event logs containing non-atomic activities with complex lifecycles.

In \cite{Bellodi.etal/CILC2010:ProbabilisticLogicBasedProM,Bellodi.etal/KSEM10:ProbabilisticDeclarativeProcessMining}, the authors present an approach for the mining of declarative process models expressed through a probabilistic logic. The approach first extract a set of integrity constraints from a log. Then,
the learned constraints are translated into Markov Logic formulas that allow for a probabilistic classification of the traces.
In \cite{Lamma.etal/BPM2007:InducingDeclarativeLogic,Chesani.etal/JPNOMC2009:ExploitingInductiveLogic}, the authors present an approach based on Inductive Logic Programming techniques to discover \gls{declare} process models. These approaches are not equipped with techniques for the analysis of the discovered models like the one presented in this paper.


In \cite{DiCiccio.Mecella/CIDM2013:TwoStepFast,DiCiccio.Mecella/ACMTMIS2015:DiscoveryDeclarativeControl}, the authors introduce a two-step algorithm for the discovery of \gls{declare} constraints. As a first step, a knowledge base is built, with information about temporal statistics gathered from logs. Then, the statistical support of constraints is computed, by querying that knowledge base. Also these works introduce a basic way to deal with redundancy based on the subsumption hierarchy of \gls{declare} templates that is non capable to deal with redundancies due to complex interactions of constraints.


In \cite{DiCiccio.etal/BPM2014:DiscoveringTargetBranched}, the authors propose an extension of the approach presented in \cite{DiCiccio.Mecella/CIDM2013:TwoStepFast,DiCiccio.Mecella/ACMTMIS2015:DiscoveryDeclarativeControl} to discover target-branched \gls{declare} constraints, i.e., constraints in which the target parameter is replaced by a disjunction of actual tasks. Here, as well as redundancy reductions based on the subsumption hierarchy of \gls{declare} constraints, also different aspects of redundancy are taken into consideration that are characteristic of target-branched \gls{declare}, such as set-dominance.
%In \cite{Raeim.etal/CoopIS2014:LogBasedUnderstanding}, the authors propose an approach based on temporal logic query checking, to discover those LTL-based business rules that are valid in the log, by checking
%against the log a (user-defined) class of rules.



%In \cite{Westergaard.Maggi/CoopIS2012:LookingintoFuture}, a semantics for Declare to consider metric temporal constraints is presented and, in \cite{DBLP:conf/bir/Maggi14}, an approach for the discovery of this type of constraints is presented.
%In \cite{Maggi.etal/BPM2013:DiscoveringDataAware}, the semantics of Declare is extended to consider conditions on the data flow as well as on the control flow. A technique based on daikon and decision trees is presented for the discovery of data-aware Declare constraints.
%In \cite{Maggi/BPMDemos2013:DeclarativeProcessMining} a set of plug-ins and tools for process mining based on declarative process models is presented.

		% !TeX root = ../Main.tex
\section{Conclusion}
\label{sec:conclusion}
%
In this work, we presented an interactive approach that takes vocal statements from the user as input and employs speech recognition to convert them into multi-perspective, declarative process models.
Our Speech2RuM approach goes beyond the state-of-the-art in text-to-constraint transformation by covering a broader range of \Declare\ templates and supporting their augmentation with data and time conditions. The integration of our approach into the RuM toolkit enables users to visualize and edit obtained models in a GUI. Furthermore, it also allows these models to directly serve as a basis for the toolkit's analysis techniques, such as conformance checking and log generation. Finally, we note that the conducted user evaluation represents the first study into the feasibility of using speech recognition for business process modeling. The results demonstrated its promising nature, although they also revealed a clear learning curve.

In future work, we aim to further develop Speech2RuM based on the obtained user feedback. The text-to-constraint component shall be improved to support less natural descriptions, e.g., those that explicitly mention the term \emph{activity} to denote a process step. Furthermore, we want to support the specification of multi-perspective constraints in a single step rather than the current two. Finally, it will be highly interesting to investigate how speech recognition can be lifted to also support the elicitation of imperative process models.
		%
		%%%%%%%%%%%%%%%%%%%%%%%%%%%%%%%%%%%%%%%%%%%%%%%%%%%%%%%%%%%%%%%%
		% Bibliography
		%%%%%%%%%%%%%%%%%%%%%%%%%%%%%%%%%%%%%%%%%%%%%%%%%%%%%%%%%%%%%%%%
		%
	%	\bibliographystyle{splncs03}
	%	\bibliography{Bibliography}
			
			\begin{thebibliography}{10}
				\providecommand{\url}[1]{\texttt{#1}}
				\providecommand{\urlprefix}{URL }
				
				\bibitem{vanderaa2019extracting}
				van~der Aa, H., Di~Ciccio, C., Leopold, H., Reijers, H.A.: Extracting
				declarative process models from natural language. In: CAiSE. pp. 365--382.
				Springer (2019)
				
				\bibitem{vanderaa2017comparing}
				van~der Aa, H., Leopold, H., Reijers, H.A.: Comparing textual descriptions to
				process models: The automatic detection of inconsistencies. Inf. Syst. 64,  447--460 (2017)
				
				\bibitem{vanderaa2018checking}
				van~der Aa, H., Leopold, H., Reijers, H.A.: Checking process compliance against
				natural language specifications using behavioral spaces. Inf. Syst.
				78,  83--95 (2018)
				
				\bibitem{vanderaa2017fragmentation}
				van~der Aa, H., Leopold, H., van~de Weerd, I., Reijers, H.A.: Causes and
				consequences of fragmented process information: Insights from a case study.
				In: {AMCIS} (2017)
				
				\bibitem{declareCSRD09}
				van~der Aalst, W.M.P., Pesic, M., Schonenberg, H.: Declarative workflows:
				Balancing between flexibility and support. CSRD  23(2),
				99--113 (2009)

				
				\bibitem{bhattacherjee2001understanding}
				Bhattacherjee, A.: Understanding information systems continuance: an
				expectation-confirmation model. MIS quarterly pp. 351--370 (2001)
				
				\bibitem{brooke1996sus}
				Brooke, J., et~al.: SUS-a quick and dirty usability scale. Usability evaluation
				in industry  189(194),  4--7 (1996)
				
				\bibitem{Burattin2015}
				Burattin, A., Maggi, F.M., Sperduti, A.: Conformance checking based on
				multi-perspective declarative process models. Expert Syst. Appl.  65,
				194--211 (2016)
				
				\bibitem{dicicio2015knowledge}
				Di~Ciccio, C., Marrella, A., Russo, A.: Knowledge-intensive processes:
				characteristics, requirements and analysis of contemporary approaches.
				JoDS 4(1),  29--57 (2015)
				
				\bibitem{dumas2013fundamentals}
				Dumas, M., La~Rosa, M., Mendling, J., Reijers, H.A., et~al.: Fundamentals of
				business process management, vol.~1. Springer (2013)
				
				
				\bibitem{friedrich2011process}
				Friedrich, F., Mendling, J., Puhlmann, F.: Process model generation from
				natural language text. In: CAiSE. pp. 482--496. Springer (2011)
				
				
				\bibitem{haisjackl2016understanding}
				Haisjackl, C., Barba, I., Zugal, S., Soffer, P., Hadar, I., Reichert, M.,
				Pinggera, J., Weber, B.: Understanding Declare models: strategies, pitfalls,
				empirical results. Software \& Systems Modeling  15(2),  325--352 (2016)
				
				\bibitem{holtzblatt2004rapid}
				Holtzblatt, K., Wendell, J.B., Wood, S.: Rapid contextual design: a how-to
				guide to key techniques for user-centered design. Elsevier (2004)
				
				\bibitem{lopez2018process}
				L{\'o}pez, H.A., Debois, S., Hildebrandt, T.T., Marquard, M.: The process
				highlighter: From texts to declarative processes and back. BPM (Demos)   2196,  66--70 (2018)
				
				\bibitem{lopez2019assisted}
				L{\'o}pez, H.A., Marquard, M., Muttenthaler, L., Str{\o}msted, R.: Assisted
				declarative process creation from natural language descriptions. In:
				EDOCW. pp. 96--99. IEEE (2019)
				
				\bibitem{Montali2010:Choreographies}
				Montali, M., Pesic, M., van~der Aalst, W.M.P., Chesani, F., Mello, P., Storari,
				S.: {Declarative Specification and Verification of Service Choreographies}.
				ACM Transactions on the Web  4(1) (2010)
				
%				\bibitem{nielsen1994enhancing}
%				Nielsen, J.: Enhancing the explanatory power of usability heuristics. In:
%				SIGCHI.
%				pp. 152--158 (1994)
				
				\bibitem{nielsen1993mathematical}
				Nielsen, J., Landauer, T.: A mathematical model of the finding of usability problems. In:
				SIGCHI.
				pp. 206--213 (1993)

				\bibitem{de2017assisting}
				de~Oliveira, J.P.M., Avila, D.T., dos Santos, R.I., Fantinato, M.: Assisting
				process modeling by identifying business process elements in natural language
				texts. In: ER Workshops, p. 154. Springer (2017)
				
				\bibitem{Pesic2007:DECLARE}
				Pesic, M., Schonenberg, H., van~der Aalst, W.M.P.: Declare: Full support for
				loosely-structured processes. In: EDOC.
				pp. 287--300 (2007)		
				
				\bibitem{sanchez2018aligning}
				S{\`a}nchez-Ferreres, J., van der Aa, H. and Carmona, J., Padr{\'o}, L.: Aligning textual and model-based process descriptions. Data Knowl. Eng. 118,  25--40 (2018)
				
				\bibitem{sanchez2019formal}
				S{\`a}nchez-Ferreres, J., Burattin, A., Carmona, J., Montali, M., Padr{\'o},
				L.: Formal reasoning on natural language descriptions of processes. In:
				BPM. pp. 86--101.
				Springer (2019)
				
				\bibitem{schumacher2013extracting}
				Schumacher, P., Minor, M., Schulte-Zurhausen, E.: Extracting and enriching
				workflows from text. In: IRI. pp. 285--292. IEEE (2013)
				
				\bibitem{selway2015formalising}
				Selway, M., Grossmann, G., Mayer, W., Stumptner, M.: Formalising natural
				language specifications using a cognitive linguistic/configuration based
				approach. Inf. Syst.  54,  191--208 (2015)
				
			\end{thebibliography}
%			

		%
	\end{sloppypar}
\end{document} 